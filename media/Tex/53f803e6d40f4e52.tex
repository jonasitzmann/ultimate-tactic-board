\documentclass[preview]{standalone}

\usepackage[english]{babel}
\usepackage[utf8]{inputenc}
\usepackage[T1]{fontenc}
\usepackage{lmodern}
\usepackage{amsmath}
\usepackage{amssymb}
\usepackage{dsfont}
\usepackage{setspace}
\usepackage{tipa}
\usepackage{relsize}
\usepackage{textcomp}
\usepackage{mathrsfs}
\usepackage{calligra}
\usepackage{wasysym}
\usepackage{ragged2e}
\usepackage{physics}
\usepackage{xcolor}
\usepackage{microtype}
\DisableLigatures{encoding = *, family = * }
\linespread{1}

\begin{document}

\begin{center}
Das Spielfeld kann auch hochkant angezeigt werden, um Platz für lange Erläuterungen zu haben.
        \begin{itemize}
         \item Die Spieler entscheiden jetzt selbst, ob sie nach rechts oder links drehen
         \item Allerdings sind ihre Bewegungen insgesamt noch unrealistisch, weil sie nicht in Blickrichtung laufen.
         \item Irgendwann sieht man hier auch ne Scheibe
        \end{itemize}
\end{center}

\end{document}
